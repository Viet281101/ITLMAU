\documentclass[12pt,a4paper,french]{article}
\usepackage[utf8]{inputenc}
\usepackage[french]{babel}
\selectlanguage{french}
\usepackage[T1]{fontenc}
\usepackage{amssymb}
\usepackage{amsmath}

\title{Compilateur purscript : rendu 1}

\begin{document}

\maketitle

\part*{Le lexer}
Le lexer est codé dans le fichier purscript\_lexer.mll et est compatible avec ocamllex.

La plupart des lexemes sont reconnus directement par la règle principale. Il y a quelques cas où l'on utilise une règle différente :

\begin{itemize}
\item les commentaires sur plusieurs lignes
\item les chaînes de caractères
\item les chaînes de caractères entre les symboles "$\backslash\backslash$"
\end{itemize}


\part*{Le parser}
Il est codé dans le fichier purscrip\_parser.mly et est compatible avec menhir.

Il reprend globalement la grammaire proposée pour le purscript. Mais doit gérer quelques cas délicats. \\

Le premier problème vient de la règle :
\[ \langle\text{tdecl}\rangle ::= \langle \text{lident}\rangle :: (\langle\text{ntype}\rangle =>)\text{* } (\langle\text{type}\rangle ->)\text{* } \langle\text{type}\rangle \]

Dans ce cas, il n'est pas possible de juste reconnaître grâce à la règle suivante :
\[  \text{list}(\text{ntype}, =>) \text{ list}(\text{type}, ->) \text{ type} \]
En effet, cela provoque un conflit car le parser ne sait pas quand passer d'une liste à l'autre.

La solution est donc de remplacer par 3 règles qui s'appellent en chaîne et qui permettent de placer nous-même les limites.
\\
Le deuxième problème vient de deux dérivations possibles dans la grammaire. En effet, si l'on souhaite reconnaitre on objet de 'type' et que l'on lit un 'uident', alors les deux dérivations suivantes sont possibles:
\[ \left[ \begin{array}{l}
	\text{type} \rightarrow \text{atype} \rightarrow \text{uident} \\
	\text{type} \rightarrow \text{ntype} \rightarrow \text{uident}
\end{array} \right.
\]
Mais pour la suite, ces deux dérivations sont équivalentes, donc nous avons forcé le parser à reconnaitre l'un des deux (en l'occurence le premier).


\part*{Le typage: point de vue général}

Pour chaque type de l'ast, on crée une fonction qui renvoie son type (par exemple typexpr)

Pour vérifier qu'un fichier est bien typé, on appelle récursivement ces différentes fonctions en vérifiant systématiquement
que les types sont bien cohérents (càd par exemple qu'on n'additionne pas 2 String)

De plus, pour les patternes, on crée la focntion ensuretyppatern qui prend en argument un patterne et un type, vérifie
que le patterne est bien cohérent avec le type et renvoie un environnement où on a ajouté les nouveaux ident apparus dans le patterne


\part*{Le typage: les environnements}

On va stocker le type de tous les ident dans des environnements locaux passées en paramètre à à peu près toutes les fonctions

Il y a un environnement global pour les types (nommé envtyps) où sont stockés les types déclarés par data, auquel s'ajoute un environnement
local qui contient les variables de type et passé en paramètre

Les fonctions et les classes sont quant à elles sont stockées dans un environnement global (nommé respectivement globalenvfonctions et envclasses)

De même que pour les types, Il y a un environnement global pour les instances (nommé globalenvinstances) où sont stockés les instances déclarées par Instance, auquel s'ajoute un environnement
local qui contient les instances passées en paramètre aux fonctions et passé en paramètre (nommé envinstances)

\part*{Le typage: gestion des instances}

Quand une fonction f liée à une classe C est utilisée, on regarde une à une toutes les instances liées à C et on prend la
première qui est cohérente avec les types des différents arguments de f

Si on n'en trouve aucune, alors on retourne une erreur.



\part*{Le typage: gestion des erreurs}

L'ast est légèrement modifiée par rapport à la grammaire fournie de manière à associer à chaque type de l'ast une position dans le fichier

Quand une erreur est détectée, on retourne cette position et on affiche le bout du fichier qui y correspond


\part*{Le typage: ce qui n'a pas été codé}

Les types d'arité >0 ne marchent pas très bien, en particulier les fichiers queens2.purs, pascal.purs et ral.purs ne typent pas

Normalement, tout le reste fonctionne.



\part*{Makefile / dune / tests}
Le projet contient également des scripts permettant d'automatiser la compilation et de tester rapidement sur de nombreux tests.

La plupart des tests sont fourni mais nous avons rajouté quelques tests afin de vérifier le comportement sur quelques cas particuliers.

Dune s'occupe de produire le fichier purscript\_main.exe. Et Make s'occupe de lancer dune, de renommer l'exécutable, et de lancer les tests.

\end{document}


